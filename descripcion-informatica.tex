\cleardoublepage
\chapter{Descripci�n inform�tica}


\section{Requisitos}
\label{sec:requisitos}

Descripci�n m�s detallada de los objetivos explicados en el cap�tulo de Objetivos

Tareas tablero Kanban del Project de GitHub

\section{Arquitectura y An�lisis}
\label{sec:arquitectura-analisis}

Como el "orquestador" se ha centrado en el formato de despliegue Master/Workers, explicaci�n de la forma de ejecuci�n de JMeter en detalle

Explicaci�n de la investigaci�n hasta que di con la forma de orquestar los servicios de manera satisfactoria para nuestro objetivo siguiendo los requisitos de JMeter.

Arquitectura inicial, con alg�n boceto de arquitectura.

Docker Swarm Architecture + GlusterFS

\section{Dise�o e Implementaci�n}
\label{sec:diseno-implementacion}

\begin{itemize}
  \item Diagrama de estados del proceso de orquestaci�n
  \item Dise�o de la API REST, funciones, endpoints...
  \item Dockerfile JMeter con su script bash
  \item Dockerfile build Golang + dep dependency manager
  \item Implementaci�n algoritmo Descubrimiento de contenedores DNSRR
\end{itemize}




\section{Pruebas}
\label{sec:pruebas}

Hablar de Cobertura de Test, como se ha organizado por paquetes, c�mo est� integrado con el Pipeline de IC de Jenkins, los triggers que tiene configurado... 
\newline

Explicar c�mo se dise�aron primero los tests b�sicos, luego la implementaci�n, y despu�s se ampli� la cobertura de tests
