\cleardoublepage
\chapter{Descripci�n inform�tica}

Descripci�n del proyecto realizado. Despu�s de unos p�rrafos introductorios el cap�tulo se divide en subcap�tulos. (de 25 a 35 p�ginas)

\section{Requisitos}
\label{sec:requisitos}

Descripci�n detallada de las funcionalidades que tendr�a que implementar la aplicaci�n (pues se asume que los requisitos se escriben antes de empezar el desarrollo). Pueden tener forma de historias de usuario o bien ser una lista de requisitos funcionales y no funcionales.


\section{Arquitectura y An�lisis}
\label{sec:arquitectura-analisis}

Descripci�n de los aspectos de alto nivel de la aplicaci�n. Diagramas de clases de an�lisis, diagramas de clases de dise�o, etc. Se debe incluir la suficiente informaci�n para que el lector pueda entender la estructura de alto nivel del software desarrollado. Se pueden incluir diagramas de casos de uso si se considera �til.


\section{Dise�o e Implementaci�n}
\label{sec:diseno-implementacion}

Descripci�n de alg�n aspecto relevante de la implementaci�n que quiera mencionarse. Por ejemplo se podr�a incluir alguno de los siguientes aspectos:
Algoritmo complejo que se haya tenido que desarrollar.
Integraci�n entre librer�as problem�tica.
Resoluci�n de alg�n bug que haya sido especialmente problem�tico.
Focalizar en alguna parte del desarrollo y describirla en m�s detalle
En esta secci�n se pueden incluir fragmentos de c�digo fuente. En este apartado se pueden incluir algunas m�tricas del proyecto (N� de clases, l�neas de c�digo, etc...). Tambi�n se puede incluir la evoluci�n del repositorio de github (gr�fico de commits por d�a).


\section{Pruebas}
\label{sec:pruebas}

En esta secci�n se describen las pruebas autom�ticas que han sido implementadas para el proyecto. Sobre los tests, conviene indicar la cobertura del c�digo. Si no se han implementado pruebas autom�ticas, deber�an haberse implementarse y describirse aqu� o tener una buena justificaci�n de por qu� no se han implementado.
