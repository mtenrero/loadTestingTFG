\cleardoublepage
\chapter{Tecnolog�as, Herramientas y Metodolog�as}

Descripci�n de los lenguajes de programaci�n, entornos de desarrrollo, herramientas auxiliares, librer�as de terceros, sistemas operativos, navegadores web, etc... utilizados para la realizaci�n del proyecto as� como la metodolog�a empleada. El grado de profundidad a la hora de explicar cada tecnolog�a depender� de lo relevante que ha sido para el proyecto y lo conocida que es. Por ejemplo, si se usa el lenguaje de programaci�n Java, no es necesario entrar en tanto detalle que si se usa un lenguaje mucho menos usado como Scala, por ejemplo. Respecto a la metodolog�a, dada la naturaleza de los proyectos, se suele describir una metodolog�a iterativa e incremental en espiral, en la que se van sucediendo reuniones con el profesor que van definiendo el �mbito del proyecto. Este cap�tulo puede tener una extensi�n entre 10 y 15 p�ginas.
