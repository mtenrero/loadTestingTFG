\cleardoublepage
\chapter{Tecnolog�as, Herramientas y Metodolog�as}

Aqu� me gustar�a hablar de los siguientes temas:



\begin{enumerate}
  \item \textbf{Lenguaje Golang:} Car�cteristicas b�sicas del lenguaje, cross-platform, cobertura de test
  \item \textbf{Git, Travis \& Jenkins:} Diferencias y features cr�ticas que me han hecho pasar de Travis a Jenkins
  \item \textbf{Swagger:}
  \item \textbf{Goa Design:} Paquete Golang, Design First API, creo que es importante por el car�cter de piensa antes de picar, muy similar a TDD
  \item \textbf{AWS} Toda la CI est� desplegada en AWS y tengo pensado montar un cluster Swarm aqu� para una futura demo.
  \item \textbf{Docker}
  \item \textbf{ElasticSearch} Caracter�sticas b�sicas
  \item \textbf{Shared Filesystem:} GlusterFS / Samba . No es Kubernetes, asi que se necesita esta configuraci�n previa para tener todos los ficheros disponibles en todo el cluster. Explicar lo que ofrece y por qu� es necesario.
  \item \textbf{Redes:} Concretamente DNS y algoritmo DNS Round Robin. Es la piedra angular en cuanto a orquestaci�n con arquitectura Master / Workers se refiere
  \item \textbf{Apache Jmeter:} Forma de ejecuci�n distribuida y arquitectura
\end{enumerate}

\section{Golang}

Go es un lenguaje de programaci�n que se comenz� a desarrollar en 2007 y naci� con el ideal de eliminar todos los obst�culos de la programaci�n actual, ya que desde hace varios a�os no hab�a salido ning�n lenguaje de programaci�n de alta importancia.

\section{DNS}

Domain Name System es un sistema de nombrado de redes IP que se utiliza para resolver nombres en direcciones IP. El servicio de resoluci�n de nombres puede tener diferentes tipos de registros. En la investigaci�n del trabajo s�lo se han usado registros de tipo \textbf{A}, los cuales traducen un nombre en una o varias direcciones IP, dependiendo de los registros A que tenga almacenados para un mismo nombre de dominio/subdominio.

\subsection{Networking: DNS Round-Robin}

DNS Round-Robin es un algoritmo de selecci�n de IP, en el que con cada petici�n que realiza el usuario, se obtiene una lista completa de todas las direcciones IP registradas en el servidor de nombres ordenada de tal forma que nunca recibe la misma IP, realizando, de esta manera, labores de balanceador de carga.

Docker implementa este algoritmo de forma alternativa en su DNS interno, y el desarrollador puede elegir si desea operar con DNS Round Robin o con \textit{Routing Mesh}, balanceo de carga interno, donde el nombre es resuelto con una �nica IP Virtual de manera no determinista.\newline

Cuando se opta por usar el modo DNS Round-Robin, por limitaciones de dise�o, resulta imposible exponer puertos hacia el host, �nicamente se permiten conexiones de red con las redes indicadas por el servicio. 