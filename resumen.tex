\chapter*{Resumen}
\addcontentsline{toc}{chapter}{Resumen} % si queremos que aparezca en el �ndice
\markboth{RESUMEN}{RESUMEN} % encabezado

Los test de rendimiento requieren de una gran potencia de c�mputo y una gran capacidad de red para poder simular de manera satisfactoria cargas de gran tama�o. Para ello es com�n ejecutar estos test de rendimiento apoy�ndose en un gran n�mero de nodos.\newline

No existen soluciones reales que simplifiquen la tediosa configuraci�n de cada uno de los nodos de una manera espec�fica para la ejecuci�n de tests.\newline

El proyecto ofrece una alternativa, un middleware sobre Docker Swarm que permite aprovechar una infraestructura cloud existente para la orquestaci�n de los contenedores especificados que actuar�n como nodos remotos para los citados tests de rendimiento. A�adiendo una capa de abstracci�n no s�lo a Docker y a la gesti�n de contenedores, sino a toda la capa de red, ya que gestiona y configura autom�ticamente las redes necesarias para aislar m�ltiples ejecuciones entre s� y entre el sistema base del host.\newline

Se ha desarrollado en \textit{Go}, ofreciendo as� la posibilidad de ser desplegado sobre diferentes Sistemas Operativos.\newline

Aunque se ha dise�ado teniendo en cuenta los test de carga, y en concreto bas�ndose en la herramienta Apache JMeter, no s�lo puede ser empleado para dichos tests. Sino que es completamente compatible con cualquier despliegue que siga el modelo maestro / esclavos.