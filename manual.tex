\section{Despliegue de ATQ}

\subsection{Binarios}

Ejecutar el binario apropiado para la plataforma:\newline

\textbf{Windows}\newline

\begin{lstlisting}[language=bash]
  > ./atq-director-win_64.exe
\end{lstlisting}  

\textbf{Unix}\newline

\begin{lstlisting}[language=bash]
  $ chmod +x atq-director-linux_64
  $ ./atq-director-linux_64
\end{lstlisting}  


\subsection{Docker}
\subsubsection{Swarm Stack}

Ejecutar el siguiente comando en un directorio que contenga el fichero \textit{docker-compose.yml} del Ap�ndice B.2

\begin{lstlisting}[language=bash]
  $ docker-compose up
\end{lstlisting}  


En caso de necesitar desplegarlo como servicio distribuido, usar el comando:


\begin{lstlisting}[language=bash]
  $ docker stack deploy --compose-file docker-compose-yml atq
\end{lstlisting}  


\section{Configuraci�n de arranque}

Durante el arranque del middleware ATQ, busca el fichero \textit{controller-config.yml} en el que se especifican una serie de configuraciones cr�ticas para el middleware.\newline

Es necesario que est� especificada la clave \textbf{port}, que define el puerto por el cual exponer la API REST y la clave \textbf{glusterpath}, que corresponde a la ruta del directorio compartido entre todos los nodos que conforman el cluster de Docker Swarm donde desplegar el middleware.\newline

En caso de no existir dicho fichero, el middleware \textbf{no arrancar�}

\section{Subida de ficheros de test}

Los ficheros de test han de subirse \textbf{antes} de lanzar una tarea sobre el middleware.\newline

La API REST est� dise�ada para que �nicamente acepte peticiones \textbf{POST} con \textit{multipart/form} en \textit{Request Header} que incluyan un par�metro \textit{file} de tipo \textit{File} y �ste incluya un fichero con la extensi�n \textit{ZIP}.\newline

En caso de no cumplir alguno de los requisitos anteriores, se devolver� un error de acuerdo a la especificaci�n de la API que se puede consultar en los ficheros OpenAPI de Swagger.\newline

Una petici�n de ejemplo tendr�a este aspecto:


\begin{lstlisting}[language=bash]
POST /api/databind/upload HTTP/1.1
Host: localhost:8080
Accept: application/atq.databind.upload+json
Content-Type: multipart/form-data; boundary=----WebKitFormBoundary7MA4YWxkTrZu0gW
Cache-Control: no-cache
Postman-Token: c1a523b2-6a04-486c-b133-e8db47d6b3d1

------WebKitFormBoundary7MA4YWxkTrZu0gW
Content-Disposition: form-data; name="file"; filename="filename.zip"
Content-Type: 


------WebKitFormBoundary7MA4YWxkTrZu0gW--
\end{lstlisting} 

\section{Creaci�n de tarea de Test v�a API}

Una vez que se dispone de un fichero de test correctamente subido al  middleware y se ha anotado su identificador, se puede proceder a crear una tarea de test.\newline

\textbf{Body}

\begin{lstlisting}[language=bash]
{
  "delay": 2,
  "master": {
    "alias": "Coordinator",
    "image": "tenrero/jmeterdistributed",
    "fileid": "1530553166",
    "environment":[
		"MODE=master",
		"TEST_PATH=/atq/data/abvetmadrid.jmx"
	],
    "replicas": 1,
    "tty": true
  },
  "name": "jm11",
  "waitCommand": {
    "command": "sleep 4",
    "expectedResult": "",
    "timeout": 2
  },
  "worker": {
    "alias": "Minions",
    "image": "tenrero/jmeterdistributed",
    "replicas": 3,
    "environment":[
		"MODE=node"
	],
    "tty": true
  }
}
\end{lstlisting}
