\cleardoublepage % empezamos en p�gina impar
\chapter{Objetivos} % t�tulo del cap�tulo (se muestra)
\label{chap:objetivos} % identificador del cap�tulo (no se muestra, es para poder referenciarlo)

El prop�sito general de este trabajo es permitir de una manera sencilla y agn�stica poder ejecutar tests de rendimiento de manera distribuida, con capacidad de crecimiento horizontal en cualquier entorno, incluso en una nube privada y con independencia del sistema operativo ya que en el mundo empresarial, es un requisito bastante demandado. \newline

Existen m�ltiples herramientas para poder ejecutar tests de rendimiento, incluso de forma distribuida, pero la mayor�a requieren una configuraci�n previa nodo a nodo, o son soluciones  SAAS, alojadas en la nube y no ofrecen ninguna alternativa on-premise. Esto fuerza a una configuraci�n manual de las aplicaciones tradicionales y obliga a tener en cuenta la configuraci�n de red entre ellos. \newline

Para poder ofrecer una funcionalidad descrita en los p�rrafos anteriores, se ha optado por apoyarse en Docker, y en su modo de funcionamiento como cl�ster, Docker Swarm, lo que ofrece una nube privada creada a partir de las m�quinas o nodos que el usuario considere, siendo posible la creaci�n de un cl�ster heterog�neo en caso de ser necesario, permitiendo una gran flexibilidad y haciendo posible el despliegue en diversas arquitecturas.\newline

Las funcionalidades principales que ofrece el framework son las siguientes:

\begin{itemize}
  \item Interfaz sencilla para el usuario (REST)
  \item Orquestaci�n de contenedores, redes y vol�menes de forma transparente 
  \item Car�cter ef�mero y reusable de la infraestructura
\end{itemize}

\clearpage

Conseguir las funcionalidades descritas conlleva cumplir una serie de objetivos intermedios:

\begin{itemize}
  \item Construir un Middleware que comunique con la API que expone el demonio de Docker
  \item Dise�ar un plan de acci�n para la orquestaci�n de contenedores
  \item Descubrir la configuraci�n de red apropiada para el car�cter de la aplicaci�n
  \item Dise�ar una API REST para el consumidor
  \item Construir un proceso de orquestaci�n de recursos Docker
\end{itemize}


Adem�s he fijado una serie de requisitos metodol�gicos para asegurar un desarrollo con la forma mas pr�ctica posible. 

\begin{itemize}
  \item Integraci�n continua desde el comienzo del proyecto
  \item Gesti�n �gil del proyecto
  \item Test-Driven-Development
\end{itemize}
