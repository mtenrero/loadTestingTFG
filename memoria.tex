%%%%%%%%%%%%%%%%%%%%%%%%%%%%%%%%%%%%%%%%%%%%%%%%%%%%%%%%%%%%%%%%%%%%%%%%%%%%%%%%
%% Plantilla de memoria en LaTeX para la ETSIT - Universidad Rey Juan Carlos
%%
%% Por Gregorio Robles <grex arroba gsyc.urjc.es>
%%     Grupo de Sistemas y Comunicaciones
%%     Escuela T�cnica Superior de Ingenieros de Telecomunicaci�n
%%     Universidad Rey Juan Carlos
%% (muchas ideas tomadas de Internet, colegas del GSyC, antiguos alumnos...
%%  etc. Muchas gracias a todos)
%%
%% La �ltima versi�n de esta plantilla est� siempre disponible en:
%%     https://github.com/gregoriorobles/plantilla-memoria
%%
%% Para obtener PDF, ejecuta en la shell:
%%   make
%% (las im�genes deben ir en PNG o JPG)

%% Adaptado para la ETSII por Marcos Tenrero <tenrero@aol.com>

%%%%%%%%%%%%%%%%%%%%%%%%%%%%%%%%%%%%%%%%%%%%%%%%%%%%%%%%%%%%%%%%%%%%%%%%%%%%%%%%

\documentclass[a4paper, 12pt]{book}
%\usepackage[T1]{fontenc}

\usepackage[a4paper, left=2.5cm, right=2.5cm, top=3cm, bottom=3cm]{geometry}
\usepackage{times}
\usepackage[latin1]{inputenc}
\usepackage[spanish]{babel} % Comenta esta l�nea si tu memoria es en ingl�s
\usepackage{url}
%\usepackage[dvipdfm]{graphicx}
\usepackage{graphicx}
\usepackage{float}  %% H para posicionar figuras
\usepackage[nottoc, notlot, notlof, notindex]{tocbibind} %% Opciones de �ndice
\usepackage{listings} %% Bloques de c�digo
\usepackage{color} %% Paquete de color para listings
\usepackage{soul} %% Formato texto
\usepackage{dirtree} %% Tree de ficheros

\graphicspath{ {./img/} }

\title{Memoria del Proyecto}
\author{Marcos Tenrero Mor�n}

\renewcommand{\baselinestretch}{1.5}  %% Interlineado

\begin{document}

\renewcommand{\refname}{Bibliograf�a}  %% Renombrando
\renewcommand{\appendixname}{Ap�ndice}

%%%%%%%%%%%%%%%%%%%%%%%%%%%%%%%%%%%%%%%%%%%%%%%%%%%%%%%%%%%%%%%%%%%%%%%%%%%%%%%%
% PORTADA

\begin{titlepage}
\begin{center}
\begin{tabular}[c]{c c}
%\includegraphics[bb=0 0 194 352, scale=0.25]{logo} &
\includegraphics[scale=0.25]{img/logo_vect.png} &
\begin{tabular}[b]{l}
\Huge
\textsf{UNIVERSIDAD} \\
\Huge
\textsf{REY JUAN CARLOS} \\
\end{tabular}
\\
\end{tabular}

\vspace{3cm}

\Large
GRADO EN INGENIER�A DE COMPUTADORES

\vspace{0.4cm}

\large
Curso Acad�mico 2017/2018

\vspace{0.8cm}

Trabajo Fin de Grado

\vspace{2.5cm}

\LARGE
DESPLIEGUE EL�STICO DE TESTS DE RENDIMIENTO

\vspace{4cm}

\large
Autor : Marcos Tenrero Mor�n \\
Tutor : Francisco Gort�zar
\end{center}
\end{titlepage}

\newpage
\mbox{}
\thispagestyle{empty} % para que no se numere esta pagina

%%%%%%%%%%%%%%%%%%%%%%%%%%%%%%%%%%%%%%%%%%%%%%%%%%%%%%%%%%%%%%%%%%%%%%%%%%%%%%%%
%%%% Dedicatoria

\chapter*{}
\pagenumbering{Roman} % para comenzar la numeracion de paginas en numeros romanos
\begin{flushright}
\textit{Dedicado a \\
mi familia / mi abuelo / mi abuela}
\end{flushright}

%%%%%%%%%%%%%%%%%%%%%%%%%%%%%%%%%%%%%%%%%%%%%%%%%%%%%%%%%%%%%%%%%%%%%%%%%%%%%%%%
%%%% Agradecimientos

\chapter*{Agradecimientos}
%\addcontentsline{toc}{chapter}{Agradecimientos} % si queremos que aparezca en el �ndice
\markboth{AGRADECIMIENTOS}{AGRADECIMIENTOS} % encabezado 


%%%%%%%%%%%%%%%%%%%%%%%%%%%%%%%%%%%%%%%%%%%%%%%%%%%%%%%%%%%%%%%%%%%%%%%%%%%%%%%%
%%%% Resumen

\chapter*{Resumen}
%\addcontentsline{toc}{chapter}{Resumen} % si queremos que aparezca en el �ndice
\markboth{RESUMEN}{RESUMEN} % encabezado

Aqu� viene un resumen del proyecto. Ha de constar de tres o cuatro p�rrafos, donde se presente de manera clara y concisa de qu� va el proyecto. 
Han de quedar respondidas las siguientes preguntas:

\begin{itemize}
  \item �De qu� va este proyecto? �Cu�l es su objetivo principal?
  \item �C�mo se ha realizado? �Qu� tecnolog�as est�n involucradas?
  \item �En qu� contexto se ha realizado el proyecto? �Es un proyecto
dentro de un marco general?
\end{itemize}

Lo mejor es escribir el resumen al final.

%%%%%%%%%%%%%%%%%%%%%%%%%%%%%%%%%%%%%%%%%%%%%%%%%%%%%%%%%%%%%%%%%%%%%%%%%%%%%%%%
%%%%%%%%%%%%%%%%%%%%%%%%%%%%%%%%%%%%%%%%%%%%%%%%%%%%%%%%%%%%%%%%%%%%%%%%%%%%%%%%
% �NDICES %
%%%%%%%%%%%%%%%%%%%%%%%%%%%%%%%%%%%%%%%%%%%%%%%%%%%%%%%%%%%%%%%%%%%%%%%%%%%%%%%%

% Las buenas noticias es que los �ndices se generan autom�ticamente.
% Lo �nico que tienes que hacer es elegir cu�les quieren que se generen,
% y comentar/descomentar esa instrucci�n de LaTeX.

%%%% �ndice de contenidos
\tableofcontents 
%%%% �ndice de figuras
\cleardoublepage
%\addcontentsline{toc}{chapter}{Lista de figuras} % para que aparezca en el indice de contenidos
\listoffigures % indice de figuras
%%%% �ndice de tablas
%\cleardoublepage
%\addcontentsline{toc}{chapter}{Lista de tablas} % para que aparezca en el indice de contenidos
%\listoftables % indice de tablas


%%%%%%%%%%%%%%%%%%%%%%%%%%%%%%%%%%%%%%%%%%%%%%%%%%%%%%%%%%%%%%%%%%%%%%%%%%%%%%%%
%%%%%%%%%%%%%%%%%%%%%%%%%%%%%%%%%%%%%%%%%%%%%%%%%%%%%%%%%%%%%%%%%%%%%%%%%%%%%%%%
% INTRODUCCI�N Y MOTIVACI�N %
%%%%%%%%%%%%%%%%%%%%%%%%%%%%%%%%%%%%%%%%%%%%%%%%%%%%%%%%%%%%%%%%%%%%%%%%%%%%%%%%

\cleardoublepage
\chapter{Introducci�n y Motivaci�n}
\label{sec:intro} % etiqueta para poder referenciar luego en el texto con ~\ref{sec:intro}
\pagenumbering{arabic} % para empezar la numeraci�n de p�gina con n�meros

Contexto en el que se enmarca el proyecto y la justificaci�n del mismo. Este cap�tulo es muy importante porque permite al lector conocer qu� sentido tiene el proyecto, qu� ofrece, por qu� es relevante su implementaci�n, los objetivos que persigue, etc. Este cap�tulo deber�a tener una extensi�n de entre 4 y 8 p�ginas.


%%%%%%%%%%%%%%%%%%%%%%%%%%%%%%%%%%%%%%%%%%%%%%%%%%%%%%%%%%%%%%%%%%%%%%%%%%%%%%%%
%%%%%%%%%%%%%%%%%%%%%%%%%%%%%%%%%%%%%%%%%%%%%%%%%%%%%%%%%%%%%%%%%%%%%%%%%%%%%%%%
% OBJETIVOS %
%%%%%%%%%%%%%%%%%%%%%%%%%%%%%%%%%%%%%%%%%%%%%%%%%%%%%%%%%%%%%%%%%%%%%%%%%%%%%%%%

\cleardoublepage % empezamos en p�gina impar
\chapter{Objetivos} % t�tulo del cap�tulo (se muestra)
\label{chap:objetivos} % identificador del cap�tulo (no se muestra, es para poder referenciarlo)

El prop�sito general de este trabajo es permitir de una manera sencilla y agn�stica poder ejecutar tests de rendimiento de manera distribuida, con capacidad de crecimiento horizontal en cualquier entorno, incluso en una nube privada y con independencia del sistema operativo ya que en el mundo empresarial, es un requisito bastante demandado. \newline

Existen m�ltiples herramientas para poder ejecutar tests de rendimiento, incluso de forma distribuida, pero la mayor�a requieren una configuraci�n previa nodo a nodo, o son soluciones  SAAS, alojadas en la nube y no ofrecen ninguna alternativa on-premise. Esto fuerza a una configuraci�n manual de las aplicaciones tradicionales y obliga a tener en cuenta la configuraci�n de red entre ellos. \newline

Para poder ofrecer una funcionalidad descrita en los p�rrafos anteriores, se ha optado por apoyarse en Docker, y en su modo de funcionamiento como cl�ster, Docker Swarm, lo que ofrece una nube privada creada a partir de las m�quinas o nodos que el usuario considere, siendo posible la creaci�n de un cl�ster heterog�neo en caso de ser necesario, permitiendo una gran flexibilidad y haciendo posible el despliegue en diversas arquitecturas.\newline

Las funcionalidades principales que ofrece el framework son las siguientes:

\begin{itemize}
  \item Interfaz sencilla para el usuario (\gls{REST})
  \item Orquestaci�n de contenedores, redes y vol�menes de forma transparente 
  \item Car�cter ef�mero y reusable de la infraestructura
\end{itemize}

\clearpage

Conseguir las funcionalidades descritas conlleva cumplir una serie de objetivos intermedios:

\begin{itemize}
  \item Construir un Middleware que comunique con la \gls{API} que expone el demonio de Docker
  \item Dise�ar un plan de acci�n para la orquestaci�n de contenedores
  \item Descubrir la configuraci�n de red apropiada para el car�cter de la aplicaci�n
  \item Dise�ar una \gls{API} \gls{REST} para el consumidor
  \item Construir un proceso de orquestaci�n de recursos Docker
\end{itemize}


Adem�s he fijado una serie de requisitos metodol�gicos para asegurar un desarrollo con la forma mas pr�ctica posible. 

\begin{itemize}
  \item Integraci�n continua desde el comienzo del proyecto
  \item Gesti�n �gil del proyecto
  \item \gls{TDD} Test-Driven-Development
\end{itemize}



%%%%%%%%%%%%%%%%%%%%%%%%%%%%%%%%%%%%%%%%%%%%%%%%%%%%%%%%%%%%%%%%%%%%%%%%%%%%%%%%
%%%%%%%%%%%%%%%%%%%%%%%%%%%%%%%%%%%%%%%%%%%%%%%%%%%%%%%%%%%%%%%%%%%%%%%%%%%%%%%%
% TECNOLOG�AS Y HERRAMIENTAS %
%%%%%%%%%%%%%%%%%%%%%%%%%%%%%%%%%%%%%%%%%%%%%%%%%%%%%%%%%%%%%%%%%%%%%%%%%%%%%%%%

\cleardoublepage
\chapter{Metodolog�as}

Este proyecto ha seguido la filosof�a de las metodolog�as �giles, es decir, un desarrollo e incremental.\newline

Dado el car�cter din�mico de la disponibilidad del tiempo, se ha optado por una planificaci�n \textbf{Kanban}, una vez definidos los requisitos y plan de acci�n del proyecto.\newline

Adem�s ha sido vital afrontar la filosof�a DevOps, aplicando desde el comienzo del proyecto la integraci�n continua, para asegurar la salud del producto con cada nueva iteraci�n, ya que al trabajar directamente con la \gls{API} del demonio de Docker, cualquier m�nimo cambio afectaba al resto del proyecto.

\section{Extreme Programming}

Extreme Programming es una pr�ctica bastante extendida en el mundo del desarrollo de software. Obliga a definir la estructura del c�digo e incluso los tests antes de desarrollar el producto final, de tal manera, en primer lugar se dise�an y se implementan los tests, los cuales fallar�n hasta que la funcionalidad est� implementada completamente.\newline

Esta metodolog�a obliga al programador a pensar en el dise�o en un primer lugar, dotando al software de mas calidad.

\section{Kanban}

Kanban es un modelo �gil de organizaci�n del desarrollo de software que se caracteriza en la divisi�n de las tareas en diferentes estados, t�picamente, \textit{PENDIENTE}, \textit{EN PROGRESO} y \textit{TERMINADO}. \newline

Se ha utilizado la gesti�n de proyectos en el propio repositorio que ofrece la plataforma GitHub\footnote{https://github.com/mtenrero/ATQ-Director/projects/1} para que estuviese disponible para la comunidad de forma sencilla. Adem�s ofrece sincronizaci�n autom�tica de tareas basada en \textit{Issues} y \textit{Pull Requests}.\newline


\begin{figure}[h]
    \centering
    \includegraphics[width=1\textwidth]{kanban-github}
    \caption{Tablero Kanban del proyecto en GitHub}
    \label{fig:kanban}
\end{figure}


%%%%%%%%%%%%%%%%%%%%%%%%%%%%%%%%%%%%%%%%%%%%%%%%%%%%%%%%%%%%%%%%%%%%%%%%%%%%%%%%
%%%%%%%%%%%%%%%%%%%%%%%%%%%%%%%%%%%%%%%%%%%%%%%%%%%%%%%%%%%%%%%%%%%%%%%%%%%%%%%%
% TECNOLOG�AS Y HERRAMIENTAS %
%%%%%%%%%%%%%%%%%%%%%%%%%%%%%%%%%%%%%%%%%%%%%%%%%%%%%%%%%%%%%%%%%%%%%%%%%%%%%%%%

\cleardoublepage
\chapter{Tecnolog�as, Herramientas y Metodolog�as}

Aqu� me gustar�a hablar de los siguientes temas:



\begin{enumerate}
  \item \textbf{Lenguaje Golang:} Car�cteristicas b�sicas del lenguaje, cross-platform, cobertura de test
  \item \textbf{Git, Travis \& Jenkins:} Diferencias y features cr�ticas que me han hecho pasar de Travis a Jenkins
  \item \textbf{Swagger:}
  \item \textbf{Goa Design:} Paquete Golang, Design First API, creo que es importante por el car�cter de piensa antes de picar, muy similar a TDD
  \item \textbf{AWS} Toda la CI est� desplegada en AWS y tengo pensado montar un cluster Swarm aqu� para una futura demo.
  \item \textbf{Docker}
  \item \textbf{ElasticSearch} Caracter�sticas b�sicas
  \item \textbf{Shared Filesystem:} GlusterFS / Samba . No es Kubernetes, asi que se necesita esta configuraci�n previa para tener todos los ficheros disponibles en todo el cluster. Explicar lo que ofrece y por qu� es necesario.
  \item \textbf{Redes:} Concretamente DNS y algoritmo DNS Round Robin. Es la piedra angular en cuanto a orquestaci�n con arquitectura Master / Workers se refiere
  \item \textbf{Apache Jmeter:} Forma de ejecuci�n distribuida y arquitectura
\end{enumerate}

\section{Golang}

Go es un lenguaje de programaci�n que se comenz� a desarrollar en 2007 \cite{Pike:2012:GG:2384716.2384720} y naci� con el ideal de eliminar todos los obst�culos de la programaci�n actual \cite{donovan2015go}, ya que desde hace varios a�os no hab�a salido ning�n lenguaje de programaci�n de alta importancia. Se necesitaba que estuviese dise�ado por completo, teniendo en cuenta factores de la inform�tica actual, como la concurrencia o la rapidez en la compilaci�n y en la codificaci�n.

Los or�genes de Go se remontan a los lenguajes Oberon 2 , C y Alef \cite{donovan2015go}. Nace como un proyecto de Google como soluci�n para la codificaci�n de soluciones complejas.

Como particular caracter�stica \cite{pike2009go}, Go es un lenguaje de programaci�n con recolector de basura, para permitir as� trabajar de una forma correcta con la concurrencia de las aplicaciones. \newline

 El compilador de Go se ide� de tal forma para que fuese compatible nativamente con todos los Sistemas Operativos, introduciendo el cross-compile (Compilaci�n para otras plataformas o arquitecturas en un �nico Sistema Operativo) como uno de sus puntos fuertes.
 
\section{Dep: Go Dependency Management}

Uno de los puntos fuertes de Go, es la gesti�n de dependencias durante la compilaci�n \cite{pike2009go}. Sin embargo, de cara al desarrollador, por defecto, carece de un fichero a nivel global de proyecto para poder definir las dependencias del mismo y/o la versi�n con la que se desea trabajar, sino que se debe especificar a trav�s de sentencias \textit{import} en los ficheros \textit{.go} del c�digo. Adem�s, para poder usar esas dependencias, es necesario que est�n presentes en el \$GOPATH del sistema.\newline

Dep nace como un experimento de Golang\footnote{https://github.com/golang/dep} , preparada para usar en producci�n, aunque sin llegar a ser la herramienta oficial de gesti�n de dependencias a nivel de proyecto/usuario. Emplea un fichero TOML en la raiz del proyecto, en �l se indica la dependencia requerida, su versi�n e incluso la rama del control de versiones desde la cual obtener los paquetes. \newline

\section{Goa Design: Design-first Network framework}

Goa\footnote{https://goa.design} es un framework completo para construir microservicios en Go que invierte la forma de construir APIs web completamente. Posee generaci�n autom�tica de c�digo y documentaci�n. \newline

Es un framework enfocado al dise�o de la API en primer lugar, y es lo que hace a este framework �nico. Posee su propio lenguaje DSL (lenguaje descriptivo) en el que antes de codificar, obliga al programador a pensar en el dise�o de la API, ya que esta debe ser definida en un primer lugar. Permite definir desde el endpoint, el contenido que consumir� y los par�metros que recibir�.

\section{OpenAPI Specification}

OpenAPI es una especificaci�n mantenida e ideada por la comunidad Open-Source en la plataforma GitHub\footnote{https://github.com/OAI/OpenAPI-Specification}, independiente de cualquier lenguaje de programaci�n, que permite definir cualquier tipo con toda la especificaci�n completa de una API, para que sea comprensible, tanto para personas como para ordenadores, ya que se basa en ficheros JSON y YAML.

\section{DNS}

Domain Name System es un sistema de nombrado de redes IP que se utiliza para resolver nombres en direcciones IP. El servicio de resoluci�n de nombres puede tener diferentes tipos de registros. En la investigaci�n del trabajo s�lo se han usado registros de tipo \textbf{A}, los cuales traducen un nombre en una o varias direcciones IP, dependiendo de los registros A que tenga almacenados para un mismo nombre de dominio/subdominio.

\subsection{DNS Round-Robin}

DNS Round-Robin es un algoritmo de selecci�n de IP, en el que con cada petici�n que realiza el usuario, se obtiene una lista completa de todas las direcciones IP registradas en el servidor de nombres ordenada de tal forma que nunca recibe la misma IP, realizando, de esta manera, labores de balanceador de carga.

Docker implementa este algoritmo de forma alternativa en su DNS interno, y el desarrollador puede elegir si desea operar con DNS Round Robin o con \textit{Routing Mesh}, balanceo de carga interno, donde el nombre es resuelto con una �nica IP Virtual de manera no determinista.\newline

Cuando se opta por usar el modo DNS Round-Robin, por limitaciones de dise�o, resulta imposible exponer puertos hacia el host, �nicamente se permiten conexiones de red con las redes indicadas por el servicio. 


%%%%%%%%%%%%%%%%%%%%%%%%%%%%%%%%%%%%%%%%%%%%%%%%%%%%%%%%%%%%%%%%%%%%%%%%%%%%%%%%
%%%%%%%%%%%%%%%%%%%%%%%%%%%%%%%%%%%%%%%%%%%%%%%%%%%%%%%%%%%%%%%%%%%%%%%%%%%%%%%%
% DESCRIPCI�N INFORM�TICA %
%%%%%%%%%%%%%%%%%%%%%%%%%%%%%%%%%%%%%%%%%%%%%%%%%%%%%%%%%%%%%%%%%%%%%%%%%%%%%%%%

\cleardoublepage
\chapter{Descripci�n inform�tica}


\section{Requisitos}
\label{sec:requisitos}

Tras un an�lisis del proyecto a desarrollar se lleg� a la conclusi�n de que hab�a que seguir un enfoque top-down, o de arriba hacia abajo. Era necesario que durante toda la etapa del desarrollo, se tuviese muy claro a d�nde se pretend�a llegar. Siendo los objetivos principales los siguientes:

\begin{itemize}
  \item Capacidad de desplegar los servicios Docker necesarios para una tarea espec�fica
  \item Networking autom�tico y transparente
  \item Interfaz sencilla para el usuario
  \item Integraci�n completa con Apache JMeter
\end{itemize}

As� se mostr� en el tablero Kanban del proyecto, proporcionado por GitHub para el repositorio concreto\footnote{https://github.com/mtenrero/ATQ-Director}.


\section{Arquitectura y An�lisis}
\label{sec:arquitectura-analisis}


Golang fue el lenguaje de programaci�n elegido debido a su alta simplicidad en cuanto a concurrencia se refiere, por permitir ser compilado y generar binarios para los mayores Sistemas Operativos y debido a la existencia de un paquete distribuido por Docker para comunicarse directamente con la API que ofrece el demonio de Docker.

La mayor complejidad presente en este proyecto ha sido obtener las direcciones virtuales de cada contenedor desplegado sobre el cluster Swarm, ya que son necesarias para poder comunicarse con las instancias directamente. Docker, por defecto, cuando crea una red entre servicios, puede invocar directamente al nombre del servicio para acceder a �l, pero en el caso de que se despligue el servicio en modo replicado y �ste haya sido configurado para tener m�s de una replica, por defecto, Docker act�a como un balanceador de carga y s�lo responde con una �nica direcci�n IP correspondiente a un �nico contenedor. Esto ha sido un duro handicap, ya que se requer�a conocer todas las direcciones de los contenedores desplegados.\newline


Durante el an�lisis de c�mo afrontar esta problem�tica, surgieron dos enfoques totalmente diferentes


\subsection{Primer enfoque: Starter en cada imagen a desplegar}

Este enfoque surge tomando como referencia la mayor�a de herramientas para service discovery usadas en Docker y en Kubernetes como Consul.io\footnote{https://www.consul.io/}. 

Cada contenedor a desplegar, tiene una imagen modificada, la cual, antes de ejecutar el entrypoint predefinido en la imagen, ejecuta una peque�a aplicaci�n, la cual, obtiene la IP virtual del contenedor y se la comunica al agente controlador, el cual lleva un listado de todas los contenedores descubiertos. 

%% INSERTAR IMAGEN CON LA ARQUITECTURA

Este enfoque, implicaba modificar todas las posibles im�genes a usar con la aplicaci�n, con lo que se reduc�a dr�sticamente la facilidad de uso con otras aplicaciones al requerir que en caso de no existir la imagen modificada deseada a desplegar con Automation Test Queue, obligar�a al usuario a modificarla por �l mismo.

\subsection{Segundo enfoque: Single Daemon y DNS Round Robin}

Con una configuraci�n de red entre servicios Docker configurada para operar con el algoritmo DNS Round Robin, se consigue disponer de la lista de todas las direcciones IP Virtuales de los contenedores de un mismo servicio.\newline

Disponiendo de esta informaci�n, dejar�a de ser necesario un descubrimiento de servicios como el expuesto en el anterior punto, dotando a la aplicaci�n de la sencillez de uso deseada debido a que el usuario ya no tendr�a que modificar la imagen a utilizar.

%% INSERTAR IMAGEN CON LA ARQUITECTURA

\section{Dise�o e Implementaci�n}
\label{sec:diseno-implementacion}

Aunque, la planificaci�n de la herramienta se ha realizado con un enfoque bottom-up, el dise�o de la misma va a ser expuesto utilizando el enfoque contrario, top-down, es decir, desde lo mas abstracto a lo m�s espec�fico. 

\subsection{API REST}

El dise�o de la interfaz con la que trabajar� el usuario que utilice esta aplicaci�n ha sido dise�ada meticulosamente utilizando el framework de Go \textit{GoaDesign}, el cual, obliga a definir el dise�o de la API con un lenguaje declarativo propio antes de la implementaci�n.\newline

Se define el concepto \textbf{Tarea} como la definici�n de las pruebas a ejecutar usando el framework Automation Test Queue.\newline 

%% INSERTAR ESTRUCTURA DE LA TAREA

Se ofrecen diferentes acciones descritas a continuaci�n

\begin{itemize}
  \item \textbf{/databind} Operaciones relacionadas con la gesti�n de archivos
  \subitem \textit{GET /list} Devuelve una lista de los ficheros disponibles en el orquestador
  \subitem \textit{POST /upload} Subir un fichero \*.zip con contenidos para que est� disponible para una futura tarea
  \item \textbf{/monitoring} Monitorizaci�n del framework
  \subitem \textit{/ping} Devuelve un HTTP 200OK si el orquestador est� operativo
  \item \textbf{/swarm} Estado del cluster Swarm
  \subitem \textit{GET /} Devuelve los detalles del Cluster Swarm
  \item \textbf{/task} Operaciones con las tareas a lanzar con el framework
      \subitem \textit{PUT /task} Crea una nueva tarea
      \subitem \textit{DELETE /task/\{id\}} Elimina una tarea ya planificada
      \subitem \textit{GET /task/\{id\}} Inspecciona una tarea planificada
\end{itemize}

\subsection{Fase de orquestaci�n}
\subsection{Im�genes Docker}
\subsection{Service Discovery}

\section{Integraci�n Continua}
\label{sec:integracion-continua}

Inicialmente, el proyecto, al estar alojado en el servicio de control de versiones de GitHub, en un repositorio de c�digo abierto, se ofrec�an diversas opciones SAAS de Integraci�n Continua, entre ellas TravisCI o CircleCI.


Al comenzar el desarrollo, las caracter�sticas que ofrec�a TravisCI encajaban perfectamente con los requisitos de Automation Test Queue:

\begin{itemize}
  \item Pipeline-as-code
  \item Enfocado a Contenedores
  \item Integraci�n con GitHub
\end{itemize}

Se utiliz� durante la mitad de desarrollo del proyecto y ofreci� muy buenos resultados. Pero lleg� un momento en el que era necesario que el host donde se ejecutaban los tests cada vez que realizaba un commit formase parte de un cluster de Docker Swarm, y esto no se pod�a conseguir con la soluci�n que ofrec�a TravisCI.\newline

La limitaci�n de TravisCI era muy importante ya que romp�a la l�nea de Integraci�n Continua, por lo que no qued� mas remedio que optar por una soluci�n mas configurable, Jenkins.

No se opt� desde un principio por Jenkins debido a la necesidad de disponer de una m�quina con acceso a la red las 24 horas al d�a y la configuraci�n del mismo.\newline

El despliegue de las m�quinas destinadas a la l�nea de integraci�n continua se realiz� sobre m�quinas EC2 de Amazon Web Services, todas ellas bajo el mismo grupo de disponibilidad y con un servicio de IP el�stica, permitiendo de esta manera que la direcci�n de acceso a Jenkins fuese est�tica. Adem�s se incluy� un registro A en el DNS de mi dominio particular para poder acceder con una direcci�n\footnote{http://atq.mtenrero.com:8080} mas f�cil de recordar a�n.

Una parte vital para el correcto funcionamiento de la l�nea de integraci�n continua fue configurar webhooks en GitHub para apuntar a la instancia de Jenkins adem�s de habilitar la integraci�n propia de Jenkins. Esto permiti� que con cada evento en el repositorio, como \textit{commits} o \textit{pull-requests} se enviase una notificaci�n a Jenkins para as� poder ejecutar el job especificado y lanzar los tests unitarios sobre todas las ramas del proyecto para asegurar la regresi�n y un buen funcionamiento del mismo.\newline


Hace apenas dos a�os, Cloudbees, la organizaci�n que es oficialmente responsable del desarrollo de Jenkins, lanz� una caracter�stica que llamaron \textit{Jenkins Declarative Pipelines} \footnote{https://jenkins.io/blog/2017/02/03/declarative-pipeline-ga/}. Estos pipelines, permiten escribir, de una forma desriptiva, las tareas y procesos a realizar cada vez que el job es ejecutado.

Esto ha permitido describir el workflow de ejecuci�n de manera totalmente agn�stica y compatible con cualquier Jenkins desplegado, siempre que tenga Docker instalado, ya que hace uso del denominado Docker-in-Docker, que permite utilizar el demonio de Docker del host dentro de un contenedor ya existente.


\section{Pruebas}
\label{sec:pruebas}

Hablar de Cobertura de Test, como se ha organizado por paquetes, c�mo est� integrado con el Pipeline de IC de Jenkins, los triggers que tiene configurado...
\newline

Explicar c�mo se dise�aron primero los tests b�sicos, luego la implementaci�n, y despu�s se ampli� la cobertura de tests



%%%%%%%%%%%%%%%%%%%%%%%%%%%%%%%%%%%%%%%%%%%%%%%%%%%%%%%%%%%%%%%%%%%%%%%%%%%%%%%%
%%%%%%%%%%%%%%%%%%%%%%%%%%%%%%%%%%%%%%%%%%%%%%%%%%%%%%%%%%%%%%%%%%%%%%%%%%%%%%%%
% CONCLUSIONES Y TRABAJOS FUTUROS%
%%%%%%%%%%%%%%%%%%%%%%%%%%%%%%%%%%%%%%%%%%%%%%%%%%%%%%%%%%%%%%%%%%%%%%%%%%%%%%%%

\cleardoublepage
\chapter{Conclusiones y trabajos futuros}
\label{chap:conclusiones-trabajos-futuros}

Me gustar�a hablar de la integraci�n completa con OS Windows, otros tipos de despliegue como en AWS de manera nativa, otros modelos diferentes a la arquitectura Master/Workers...

Quiz�s de una interfaz web completa con Angular o similar... 


%%%%%%%%%%%%%%%%%%%%%%%%%%%%%%%%%%%%%%%%%%%%%%%%%%%%%%%%%%%%%%%%%%%%%%%%%%%%%%%%
%%%%%%%%%%%%%%%%%%%%%%%%%%%%%%%%%%%%%%%%%%%%%%%%%%%%%%%%%%%%%%%%%%%%%%%%%%%%%%%%
% BIBLIOGRAFIA %
%%%%%%%%%%%%%%%%%%%%%%%%%%%%%%%%%%%%%%%%%%%%%%%%%%%%%%%%%%%%%%%%%%%%%%%%%%%%%%%%

\cleardoublepage

% Las siguientes dos instrucciones es todo lo que necesitas
% para incluir las citas en la memoria
\bibliographystyle{abbrv}
\bibliography{memoria}  % memoria.bib es el nombre del fichero que contiene
% las referencias bibliogr�ficas. Abre ese fichero y mira el formato que tiene,
% que se conoce como BibTeX. Hay muchos sitios que exportan referencias en
% formato BibTeX. Prueba a buscar en http://scholar.google.com por referencias
% y ver�s que lo puedes hacer de manera sencilla.
% M�s informaci�n: 
% http://texblog.org/2014/04/22/using-google-scholar-to-download-bibtex-citations/


%%%%%%%%%%%%%%%%%%%%%%%%%%%%%%%%%%%%%%%%%%%%%%%%%%%%%%%%%%%%%%%%%%%%%%%%%%%%%%%%
%%%%%%%%%%%%%%%%%%%%%%%%%%%%%%%%%%%%%%%%%%%%%%%%%%%%%%%%%%%%%%%%%%%%%%%%%%%%%%%%
% AP�NDICE(S) %
%%%%%%%%%%%%%%%%%%%%%%%%%%%%%%%%%%%%%%%%%%%%%%%%%%%%%%%%%%%%%%%%%%%%%%%%%%%%%%%%

\cleardoublepage
\appendix
\chapter{Manual de usuario}
\label{app:manual}

\appendixname{b}
\chapter{Fragmentos de c�digo}
\label{code-snippets}

\definecolor{pblue}{rgb}{0.13,0.13,1}
\definecolor{pgreen}{rgb}{0,0.5,0}
\definecolor{pred}{rgb}{0.9,0,0}
\definecolor{pgrey}{rgb}{0.46,0.45,0.48}

\lstset{language=Java,
  showspaces=false,
  showtabs=false,
  breaklines=true,
  showstringspaces=false,
  breakatwhitespace=true,
  commentstyle=\color{pgreen},
  keywordstyle=\color{pblue},
  stringstyle=\color{pred},
  basicstyle=\small,
  moredelim=[il][\textcolor{pgrey}]{$$},
  moredelim=[is][\textcolor{pgrey}]{\%\%}{\%\%}
}

\begin{lstlisting}[language=Java, caption=Jenkisfile CI Docker-in-Docker]
pipeline {

    agent {
        docker {
            image 'tenrero/golang-dep-alpine:1.10.2'
            reuseNode true
            args '-it -v /var/run/docker.sock:/var/run/docker.sock -v $WORKSPACE:/tmp/app'
        }
    }

    stages {        
        stage('Prepare Environment') {
            steps {
                sh 'echo $GOPATH'
                sh 'mkdir -p /go/src/github.com/mtenrero/'
                sh 'ln -s /tmp/app /go/src/github.com/mtenrero/ATQ-Director'
                sh 'ls -a /go/src/github.com/mtenrero/ATQ-Director'
                sh 'go get -u github.com/golang/dep/cmd/dep'
                sh 'go get -u github.com/golang/lint/golint'
                sh 'go get -u github.com/tebeka/go2xunit'
                sh 'go get -u golang.org/x/tools/cmd/cover'
                sh 'go get -u github.com/mattn/goveralls'
            }
        }

        stage('Download Vendor') {
            steps {
                sh 'cd /go/src/github.com/mtenrero/ATQ-Director && dep ensure'
            }
        }

        stage('Test') {
            steps {
                sh 'cd /go/src/github.com/mtenrero/ATQ-Director && go test ./... -race -coverprofile=coverage.txt -covermode=atomic'
            }
        }

        stage('Build') {
            steps {
                sh 'cd /go/src/github.com/mtenrero/ATQ-Director && ./build.sh'
            }
        }
    }
    
}
\end{lstlisting}


\end{document}
